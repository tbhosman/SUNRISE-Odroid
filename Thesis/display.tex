\subsection{Local display}\label{sec:display}
\subsubsection{Display data file}
To display data on a local webpage \ref{eis:1.4} \ref{eis:1.5}, it was decided that it would be the most convenient to write data to a text file, which would then be read by a script that controls the webpage. Because this read script would most likely be a JavaScript file, it was thought that the text file should be of the .json format. The code that is used to print this .json file can be found in Script \ref{print_json}.

\scriptsize
\begin{lstlisting}[language=C,caption={Writing a .json file for the local webpage},label={print_json}]
FILE *f = fopen("localDisplayData.json", "w");
if (f == NULL){
	printf("Error opening file!\n");
} else {
	fprintf(f, "{");
	fprintf(f, "\"temperature\": %f,\n", weatherFinalValues[5]);
	fprintf(f, "\"wind-chill\": %f,\n", weatherFinalValues[6]);
	fprintf(f, "\"humidity\": %f,\n", weatherFinalValues[7]);
	fprintf(f, "\"wind speed\": %f,\n", weatherFinalValues[8]);
	fprintf(f, "\"radiation\": %f,\n", weatherFinalValues[9]);
	fprintf(f, "\"air pressure\": %f,\n", weatherFinalValues[10]);
	fprintf(f, "\"wind direction\": %f\n", weatherFinalValues[11]);
	fprintf(f, "}");
	fclose(f);
}
\end{lstlisting}
\normalsize

\subsubsection{Testing the local webpage}
A simple webpage was made to ensure that running the local webpage by reading the text file is possible. A problem could arise, for example, because the two programs are both accessing the same file.\\

The JavaScript file can be found in Script \ref{script:JavaScript_test}. The script makes use of \textit{Ajax} \cite{jquery} to asynchronously refresh the webpage, so it will refresh the data without refreshing the entire webpage. It will read the \verb|localDisplayData.json| file every second (line 11) and will write this information to an HTML file (see Script \ref{script:HTML_test}). This writing is done in line 4 to 6 of Script \ref{script:JavaScript_test}.\\

\scriptsize
	\lstinputlisting [language=Java,caption=Test code for reading the .json file,label=script:JavaScript_test] {code/LocalDisplay.js}
	\lstinputlisting [language=HTML,caption=HTML file for displaying the data of the .json file,label=script:HTML_test] {code/LocalDisplayTest.html}
\normalsize

\subsubsection{WebGL}
The webpage, designed by another team within project SUNRISE, makes use of WebGL to display the information generated by the ODROID. A guide made by Hardkernel was used to install WebGL on the ODROID \cite{WebGL_guide}, after which a tip on their forum was used to enable WebGL in Chromium \cite{WebGL_forum}.\\

At first, these solutions, including many others, were unable to get WebGL running on the ODROID, eventhough the ODROID should be able to run WebGL. A lot of time was spent trying to fix this, after which it was decided that a good option might be to simply use a fresh install of the operating system. This was done by using a new ODROID C1+ (so the previous ODROID could still be used as a backup). This device was able to run WebGL after using the earlier mentioned sources. It is unclear why this was the case, but it is thought that the previous ODROID was either partly broken or had some setting disabled by a previous developer.\\

During the configuration of the new ODROID, it was decided that it would be convenient for future development to make an instruction file containing all elements that need to be installed and configured to be able to run all the systems. This instruction can be found in Appendix \ref{app:setup_manual}.

\subsubsection{Remote desktop}
The ODROID should be accessable remotely because it is not easily reachable when it is integrated in the charging station \ref{eis:2.1}. There are two elements required for this to work: a VPN connection to the router in the station and a remote desktop client. For the VPN connection, the OpenVPN  client \cite{openvpn} was used. This required VPN files that can be downloaded from the router.\\

During the first setup of the VPN this solution did not work. It took some time to figure out that the downloaded files were not configured correctly and needed to be edited before they could be used. A more detailed description can be found in the setup manual (see Appendix \ref{app:setup_manual}).\\

Finally, ``Remote Desktop Connection'' (installed default on Windows) was used to connect to the ODROID. For the ODROID to facilitate this connection, it needs a couple of packages. This too is described in detail in the setup manual (see Appendix \ref{app:setup_manual}).