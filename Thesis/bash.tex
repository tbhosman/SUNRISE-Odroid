\subsection{Bash}\label{sec:bash}
For the system to correctly function, a couple of scripts had to be designed in Bash. These scripts take part in ensuring that the system remains functioning \ref{eis:2.3} or is able to recover after a power failure \ref{eis:2.2}. All scripts are booted on startup of the desktop using the \textit{Upstart} \cite{upstart} package. A \verb|README| file was made on the desktop of the ODROID to describe the scripts and holds information about how to turn them off (See Appendix \ref{app:bash_readme}). This was done to ensure easy future use.

\subsubsection{Xvnc killer}
One of the first observations that was done when the remote desktop functionality was working, was that the ODROID was significantly slower whenever someone connected. This is as expected, because the ODROID simply has more information to process. However, this issue remained even when all remote desktops were closed. It became an even greater issue when the ODROID could not handle all the processing anymore, after which it would completely freeze for up to an hour.\\

It was later discovered that a program called \verb|Xvnc| would boot whenever a remote desktop connected, and would not shut down when the user disconnected. A Bash script was therefore made that would count the number of open \verb|Xvnc| processes, and that would kill the oldest if it exceeded a certain amount (See Script \ref{script:xvnc}). This was done in a similar way for a processes called \verb|ssh-agent|, which booted in conjuction with \verb|Xvnc|. \verb|Xvnc_killer.sh| is booted on startup using an \verb|Upstart| file (See Appendix \ref{app:xvnc_killer_conf}).

\scriptsize
	\lstinputlisting [language=Bash,caption=Xvnc\_killer.sh,label=script:xvnc] {code/Xvnc_killer.sh}
\normalsize

The script uses \verb|ps aux| to list all the processes, \verb|grep| to search for a specific process, \verb|grep -v| \verb|'<defunct>'| to remove zombies (if a process did not shut down correctly it remains in the processes list without using any memory) and \verb|-gt| to check if the search returns a value greater than 4. If there are more than 3 clients connected (the search function also finds itself as a process), it will kill the oldest running client. This is done by sorting by the time the process booted by using \verb|sort -n -k 9| (this will sort by the 9th column: start time), picking the process ID column by using \verb|awk '{print $2}'| (this prints column 2: process IDs) and then picking the oldest process using \verb|head -1|. The \verb|kill| command kills the process that has this ID. Finally, the script sleeps for a minute, after which it repeats the cycle.

\subsubsection{Browser booter}
The browser booter was designed to automatically boot a browser for the local display \ref{eis:1.4}. The display should only show the local webpage, nothing else. This implies that it has to boot full-screen and that it should not display a scroll bar nor a mouse. An \textit{Upstart} file was made to boot the Chromium browser (See Appendix \ref{app:browser_boot}). This boot uses two flags: \verb|--use-gl=egl| to enable WebGL features used in the local webpage and \verb|--kiosk| to boot full-screen. A bash script was made to start the \textit{Unclutter} \cite{unclutter} package on startup, which was set to hide the mouse after 5 seconds of inactivity (see Appendix \ref{app:hide_mouse} and \ref{app:hide_mouse_conf}). 

\subsubsection{C-code booter}
The C-code booter was made to boot the C-code on startup (see Appendix \ref{app:odroid_boot} and \ref{app:odroid_boot_conf}). This can be useful when the system has been troubled by a powercut \ref{eis:2.2}. The booter will also check if the program remains running, and it will reboot whenever it finds that it is not. Several flags in the command will ensure that the terminal does open (using \verb|-hold|), but not in front of the browser (using \verb|-iconic|). The \verb|su| command is used because the code requires a SuperUser (because of the control of GPIO pins).